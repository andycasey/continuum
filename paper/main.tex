% Copyright 2023 Andy Casey (Monash) and friends
% TeX magic by David Hogg (NYU)

\documentclass[modern]{aastex631}
\usepackage[utf8]{inputenc}
\usepackage{amsmath}

\renewcommand{\twocolumngrid}{}
\addtolength{\topmargin}{-0.35in}
\addtolength{\textheight}{0.6in}
\setlength{\parindent}{3.5ex}
\renewcommand{\paragraph}[1]{\medskip\par\noindent\textbf{#1}~---}

% figure setup
\usepackage{graphicx}
\usepackage{xcolor}
\usepackage[framemethod=tikz]{mdframed}
\usetikzlibrary{shadows}
\definecolor{captiongray}{HTML}{555555}
\mdfsetup{%
innertopmargin=2ex,
innerbottommargin=1.8ex,
linecolor=captiongray,
linewidth=0.5pt,
roundcorner=1pt,
shadow=false,
}
\newlength{\figurewidth}
\setlength{\figurewidth}{0.75\textwidth}

% text macros
\shorttitle{Stellar continuum modelling}
\shortauthors{Casey}
\newcommand{\documentname}{\textsl{Article}}
\newcommand{\sectionname}{Section}

% math macros
\newcommand{\unit}[1]{\mathrm{#1}}
\newcommand{\mps}{\unit{m\,s^{-1}}}
\newcommand{\kmps}{\unit{km\,s^{-1}}}

% notes

\sloppy\sloppypar\raggedbottom\frenchspacing
\begin{document}

\title{\Huge Stellar continuum modelling}

\author[0000-0003-0174-0564]{Andrew R. Casey}
\affiliation{School of Physics \& Astronomy, Monash University}
\affiliation{Centre of Excellence for Astrophysics in Three Dimensions (ASTRO-3D)}

%\author[0000-0003-2866-9403]{David W. Hogg}
%\affiliation{Center for Cosmology and Particle Physics, Department of Physics, New York University}
%\affiliation{Max-Planck-Institut f\"ur Astronomie, Heidelberg}
%\affiliation{Flatiron Institute, a division of the Simons Foundation}


\begin{abstract}\noindent
Continuum normalization is a common step when analyzing stellar spectra.
- various bespoke methods
- best fit simultaneously with the stellar spectrum
- that requires a model of the stellar spectrum, which can be expensive to compute
- 
\end{abstract}

\keywords{Some --- keywords --- here}

\section*{}\clearpage
\section{Introduction}\label{sec:intro}


Code is available at \url{https://github.com/andycasey/continuum} and is registered in the Python Package Index as \texttt{stellar-continuum}.

\paragraph{Software}
\texttt{numpy} \citep{numpy} ---
\texttt{matplotlib} \citep{matplotlib} ---
\texttt{scipy} \citep{scipy}.

\paragraph{Acknowledgements}
It is a pleasure to thank
% All these people are possible co-authors
    Adam Wheeler (Ohio State University),
    David W. Hogg (New York University),
    Megan Bedell (Flatiron Institute),
.
% include bibliography
\bibliographystyle{aasjournal}
\bibliography{bibliography}

\end{document}
